\section{Conditions}
The final section of the file defines parameters and boundary conditions which
define the nature of the problem to be solved. These are enclosed in the
\inltt{CONDITIONS} tag.

\subsection{Parameters}

Parameters may be required by a particular solver (for instance time-integration
parameters or solver-specific parameters), or arbitrary and only used within the
context of the session file (e.g. parameters in the definition of an initial
condition). All parameters are enclosed in the \inltt{PARAMETERS} XML element.

\begin{lstlisting}[style=XMLStyle] 
<PARAMETERS>
    ...
</PARAMETERS>
\end{lstlisting}

A parameter may be of integer or real type and may reference other parameters
defined previous to it. It is expressed in the file as

\begin{lstlisting}[style=XMLStyle]
<P> [PARAMETER NAME] = [PARAMETER VALUE] </P>
\end{lstlisting}

For example,

\begin{lstlisting}[style=XMLStyle]
<P> NumSteps = 1000              </P>
<P> TimeStep = 0.01              </P>
<P> FinTime  = NumSteps*TimeStep </P>
\end{lstlisting}

\subsection{Solver Information}

These specify properties to define the actions specific to solvers, typically
including the equation to solve, the projection type and the method of time
integration. The property/value pairs are specified as XML attributes. For
example, 
\begin{lstlisting}[style=XMLStyle] 
<SOLVERINFO>
  <I PROPERTY="EQTYPE"                VALUE="UnsteadyAdvection"    /> 
  <I PROPERTY="Projection"            VALUE="Continuous"           /> 
  <I PROPERTY="TimeIntegrationMethod" VALUE="ClassicalRungeKutta4" />
</SOLVERINFO>
\end{lstlisting}

The list of available solvers in Nektar++ can be found in
Chapter~\ref{c:solvers}.

\subsubsection{Drivers}
Drivers are defined under the \inltt{CONDITIONS} section as properties of the \inltt{SOLVERINFO} XML element. The role of a driver is to manage the solver execution from an upper level. The default driver is called \inltt{Standard} and executes the following steps:
\begin{enumerate}
\item Prints out on screen a summary of all the conditions defined in the input file.
\item Sets up the initial and boundary conditions.
\item Calls the solver defined by \inltt{SolverType}  in the \inltt{SOLVERINFO} XML element.
\item Writes the results in the output (.fld) file.
\end{enumerate}

In the following example, the driver \inltt{Standard} is used to manage the execution of the incompressible Navier-Stokes equations:

\begin{lstlisting}[style=XMLStyle]
<SOLVERINFO>
    <I PROPERTY="EQTYPE" VALUE="UnsteadyNavierStokes" />
    <I PROPERTY="SolverType" VALUE="VelocityCorrectionScheme" />
    <I PROPERTY="Projection" VALUE="Galerkin" />
    <I PROPERTY="TimeIntegrationMethod" VALUE="IMEXOrder2" />
    <I PROPERTY="Driver" VALUE="Standard" />
</SOLVERINFO>
\end{lstlisting}

If no driver is specified in the session file, the driver \inltt{Standard} is called by default. But other drivers can be used. As described in Sec. \ref{SectionIncNS_SolverInfo} and  \ref{SectionIncNS_SolverInfo_Stab}, the other possibilities are:
\begin{itemize}
\item \inltt{SteadyState} - uses the Selective Frequency Damping method (see Sec. \ref{SectionSFD}) to obtain a steady-state solution of the Navier-Stokes equations (compressible or incompressible).
\item \inltt{ModifiedArnoldi}  - computes of the leading eigenvalues and eigenmodes using modified Arnoldi method.
\item \inltt{Arpack} - computes of eigenvalues/eigenmodes using Implicitly Restarted Arnoldi Method (ARPACK).
\end{itemize}


\subsection{Variables}

These define the number (and name) of solution variables. Each variable is
prescribed a unique ID. For example a two-dimensional flow simulation may define
the velocity variables using

\begin{lstlisting}[style=XMLStyle]
<VARIABLES>
  <V ID="0"> u </V>
  <V ID="1"> v </V>
</VARIABLES>
\end{lstlisting}

\subsection{Global System Solution Information}

This section allows you to specify the global system solution parameters which
is particularly useful when using an iterative solver. An example of this
section is as follows:

\begin{lstlisting}[style=XMLStyle]
<GLOBALSYSSOLNINFO>
  <V VAR="u,v,w">
    <I PROPERTY="GlobalSysSoln"    VALUE="IterativeStaticCond" />
    <I PROPERTY="Preconditioner"   VALUE="LowEnergyBlock"/>
    <I PROPERTY="IterativeSolverTolerance"  VALUE="1e-8"/>
  </V>
  <V VAR="p">
    <I PROPERTY="GlobalSysSoln"    VALUE="IterativeStaticCond" />
    <I PROPERTY="Preconditioner"   VALUE="FullLinearSpaceWithLowEnergyBlock"/>
    <I PROPERTY="IterativeSolverTolerance"  VALUE="1e-6"/>
  </V>
</GLOBALSYSSOLNINFO>
\end{lstlisting}

The above section specifies that the global solution system for the variables
"u,v,w" should use the iIerativeStaticCond approach with the LowEnergyBlock
preconditioned and an iterative tolerance of 1e-6.  Where as the variable "p"
which also is solved with the IterativeStaticCond approach should use the
FullLinearSpaceWithLowEnergyBlock and an iterative tolerance of 1e-8.

Other parameters which can be specified include SuccessiveRHS. 

The parameters in this section override those specified in the Parameters section. 

\subsection{Boundary Regions and Conditions}

Boundary conditions are defined by two XML elements. The first defines the
various boundary regions in the domain in terms of composite entities from the
\inltt{GEOMETRY} section of the file. Each boundary region has a unique ID and
are defined as, for example,

\begin{lstlisting}[style=XMLStyle]
<BOUNDARYREGIONS>
  <B ID="0"> C[2] </B>
  <B ID="1"> C[3] </B>
</BOUNDARYREGIONS>
\end{lstlisting}

The second defines the actual boundary condition to impose on that composite
region for each of the defined solution variables, and has the form,

\begin{lstlisting}[style=XMLStyle] 
<BOUNDARYCONDITIONS>
  <REGION REF="0">
    <D VAR="u" VALUE="sin(PI*x)*cos(PI*y)" /> <D VAR="v"
    VALUE="sin(PI*x)*cos(PI*y)" />
  </REGION>
</BOUNDARYCONDITIONS>
\end{lstlisting}

Boundary condition specifications may refer to any parameters defined in the
session file. The REF attribute corresponds to a defined boundary region. The
tag used for each variable specifies the type of boundary condition to enforce.
These can be either
\begin{itemize}
    \item \inltt{D} Dirichlet 
    \item \inltt{N} Neumann 
    \item \inltt{R} Robin 
    \item \inltt{P} Periodic
\end{itemize}

% TODO: document user defined types, etc

Time-dependent boundary conditions may be specified through setting the
\inltt{USERDEFINEDTYPE} attribute. For example,

\begin{lstlisting}[style=XMLStyle]
<D VAR="u" USERDEFINEDTYPE="TimeDependent" VALUE="sin(PI*(x-t))" />
\end{lstlisting}

Periodic boundary conditions reference the corresponding boundary region with
which to enforce periodicity.

The following example provides an example of three boundary conditions for a
two-dimensional flow,

\begin{lstlisting}[style=XMLStyle]
<BOUNDARYCONDITIONS>
  <REGION REF="0">
    <D VAR="u" USERDEFINEDTYPE="TimeDependent" 
               VALUE="-cos(x)*sin(y)*exp(-2*t*Kinvis)" />
    <D VAR="v" USERDEFINEDTYPE="TimeDependent" 
               VALUE="sin(x)*cos(y)*exp(-2*t*Kinvis)" />
    <P VAR="p" VALUE=[2]/>
  </REGION>
  <REGION REF="1">
    <D VAR="u" USERDEFINEDTYPE="TimeDependent" 
               VALUE="-cos(x)*sin(y)*exp(-2*t*Kinvis)" />
    <D VAR="v" USERDEFINEDTYPE="TimeDependent" 
               VALUE="sin(x)*cos(y)*exp(-2*t*Kinvis)" />
    <N VAR="p" USERDEFINEDTYPE="H" VALUE="0.0"/>
  </REGION>
  <REGION REF="2">
    <D VAR="u" USERDEFINEDTYPE="TimeDependent" 
               VALUE="-cos(x)*sin(y)*exp(-2*t*Kinvis)" />
    <D VAR="v" USERDEFINEDTYPE="TimeDependent" 
               VALUE="sin(x)*cos(y)*exp(-2*t*Kinvis)" />
    <P VAR="p" VALUE=[0]/>
  </REGION>
  <REGION REF="3">
    <D VAR="u" USERDEFINEDTYPE="TimeDependent" 
               VALUE="-cos(x)*sin(y)*exp(-2*t*Kinvis)" />
    <D VAR="v" USERDEFINEDTYPE="TimeDependent" 
               VALUE="sin(x)*cos(y)*exp(-2*t*Kinvis)" />
    <D VAR="p" USERDEFINEDTYPE="TimeDependent" 
               VALUE="-0.25*(cos(2*x)+cos(2*y))*exp(-4*t*Kinvis)"/>
  </REGION>
</BOUNDARYCONDITIONS>
\end{lstlisting}

where the boundary regions which are periodic are linked via their region
identifier (Region 0 and Region 2).

Boundary conditions can also be loaded from file, here an example from the
Incompressible Navier-Stokes cases,

\begin{lstlisting}[style=XMLStyle]
<REGION REF="1">
  <D VAR="u" FILE="Test_ChanFlow2D_bcsfromfiles_u_1.bc" />
  <D VAR="v" VALUE="0" />
  <N VAR="p" USERDEFINEDTYPE="H" VALUE="0" />
</REGION>
\end{lstlisting}

\subsection{Functions}

Finally, multi-variable functions such as initial conditions and analytic
solutions may be specified for use in, or comparison with, simulations. These
may be specified using expressions (\inltt{<E>}) or imported from a file
(\inltt{<F>}) using the Nektar++ FLD file format

\begin{lstlisting}[style=XMLStyle]
<FUNCTION NAME="ExactSolution">
  <E VAR="u" VALUE="sin(PI*x-advx*t))*cos(PI*(y-advy*t))" />
</FUNCTION>
<FUNCTION NAME="InitialConditions">
  <F VAR="u" FILE="session.rst" />
</FUNCTION>
\end{lstlisting}

A restart file is a solution file (in other words an .fld renamed as .rst) where
the field data is specified. The expansion order used to generate the .rst file
must be the same as that for the simulation. The filename must be specified
relative to the location of the .xml file.

Other examples of this input feature can be the insertion of a forcing term,

\begin{lstlisting}[style=XMLStyle]
<FUNCTION NAME="BodyForce">
  <E VAR="u" VALUE="0" />
  <E VAR="v" VALUE="0" />
</FUNCTION>
<FUNCTION NAME="Forcing">
  <E VAR="u" VALUE="-(Lambda + 2*PI*PI)*sin(PI*x)*sin(PI*y)" />
</FUNCTION>
\end{lstlisting}

or of a linear advection term

\begin{lstlisting}[style=XMLStyle]
<FUNCTION NAME="AdvectionVelocity">
  <E VAR="Vx" VALUE="1.0" />
  <E VAR="Vy" VALUE="1.0" />
  <E VAR="Vz" VALUE="1.0" />
</FUNCTION>
\end{lstlisting}

\subsubsection{Remapping variable names}

Note that it is sometimes the case that the variables being used in the solver
do not match those saved in the FLD file. For example, if one runs a
three-dimensional incompressible Navier-Stokes simulation, this produces an FLD
file with the variables \inltt{u}, \inltt{v}, \inltt{w} and \inltt{p}. If we
wanted to use this velocity field as input for an advection velocity, the
advection-diffusion-reaction solver expects the variables \inltt{Vx}, \inltt{Vy}
and \inltt{Vz}.

We can manually specify this mapping by adding a colon to the

\begin{lstlisting}[style=XMLStyle]
<FUNCTION NAME="AdvectionVelocity">
  <F VAR="Vx,Vy,Vz" FILE="file.fld:u,v,w" />
</FUNCTION>
\end{lstlisting}

There are some caveats with this syntax:

\begin{itemize}
  \item You must specify the same number of fields for both the variable, and
  after the colon. For example, the following is not valid.
  \begin{lstlisting}[style=XMLStyle,gobble=4]
    <FUNCTION NAME="AdvectionVelocity">
      <F VAR="Vx,Vy,Vz" FILE="file.fld:u" />
    </FUNCTION>\end{lstlisting}
  \item This syntax is not valid with the wildcard operator \inltt{*}, so one
  cannot write for example:
  \begin{lstlisting}[style=XMLStyle,gobble=4]
    <FUNCTION NAME="AdvectionVelocity">
      <F VAR="*" FILE="file.fld:u,v,w" />
    </FUNCTION>
  \end{lstlisting}
\end{itemize}

\subsubsection{Time-dependent file-based functions}

With the additional argument \inltt{TIMEDEPENDENT="1"}, different files can be
loaded for each timestep. The filenames are defined using
\href{http://www.boost.org/doc/libs/1_56_0/libs/format/doc/format.html#syntax}{boost::format
  syntax} where the step time is used as variable. For example, the function
\inltt{Baseflow} would load the files \inltt{U0V0\_1.00000000E-05.fld},
\inltt{U0V0\_2.00000000E-05.fld} and so on.

\begin{lstlisting}[style=XMLStyle]
<FUNCTION NAME="Baseflow">
  <F VAR="U0,V0" TIMEDEPENDENT="1" FILE="U0V0_%14.8R.fld" />
</FUNCTION>
\end{lstlisting}

Section~\ref{sec:xml:analytic-expressions} provides the list of acceptable
mathematical functions and other related technical details.

\subsection{Quasi-3D approach}

To generate a Quasi-3D appraoch with Nektar++ we only need to create a 2D or a
1D mesh, as reported above, and then specify the parameters to extend the
problem to a 3D case. For a 2D spectral/hp element problem, we have a 2D mesh
and along with the parameters we need to define the problem (i.e. equation type,
boundary conditions, etc.). The only thing we need to do, to extend it to a
Quasi-3D approach, is to specify some additional parameters which characterise
the harmonic expansion in the third direction. First we need to specify in the
solver information section that that the problem will be extended to have one
homogeneouns dimension; here an example

\begin{lstlisting}[style=XMLStyle]
<SOLVERINFO>
  <I PROPERTY="SolverType" VALUE="VelocityCorrectionScheme"/>
  <I PROPERTY="EQTYPE" VALUE="UnsteadyNavierStokes"/>
  <I PROPERTY="AdvectionForm" VALUE="Convective"/>
  <I PROPERTY="Projection" VALUE="Galerkin"/>
  <I PROPERTY="TimeIntegrationMethod" VALUE="IMEXOrder2"/>
  <I PROPERTY="HOMOGENEOUS" VALUE="1D"/>
</SOLVERINFO>
\end{lstlisting}

then we need to specify the parameters which define the 1D harmonic expanson
along the z-axis, namely the homogeneous length (LZ) and the number of modes in
the homogeneous direction (HomModesZ). HomModesZ corresponds also to the number
of quadrature points in the homogenous direction, hence on the number of 2D
planes discretized with a specral/hp element method.

\begin{lstlisting}[style=XMLStyle]
<PARAMETERS>
  <P> TimeStep      = 0.001   </P>
  <P> NumSteps      = 1000    </P>
  <P> IO_CheckSteps = 100     </P>
  <P> IO_InfoSteps  = 10      </P>
  <P> Kinvis        = 0.025   </P>
  <P> HomModesZ     = 4       </P>
  <P> LZ            = 1.0     </P>
</PARAMETERS>
\end{lstlisting}

In case we want to create a Quasi-3D approach starting form a 1D spectral/hp
element mesh, the procedure is the same, but we need to specify the parameters
for two harmonic directions (in Y and Z direction). For Example,

\begin{lstlisting}[style=XMLStyle]
<SOLVERINFO>
  <I PROPERTY="EQTYPE" VALUE="UnsteadyAdvectionDiffusion" />
  <I PROPERTY="Projection" VALUE="Continuous"/>
  <I PROPERTY="HOMOGENEOUS" VALUE="2D"/>
  <I PROPERTY="DiffusionAdvancement" VALUE="Implicit"/>
  <I PROPERTY="AdvectionAdvancement" VALUE="Explicit"/>
  <I PROPERTY="TimeIntegrationMethod" VALUE="IMEXOrder2"/>
</SOLVERINFO>
<PARAMETERS>
  <P> TimeStep      = 0.001 </P>
  <P> NumSteps      = 200   </P>
  <P> IO_CheckSteps = 200   </P>
  <P> IO_InfoSteps  = 10    </P>
  <P> wavefreq      = PI    </P>
  <P> epsilon       = 1.0   </P>
  <P> Lambda        = 1.0   </P>
  <P> HomModesY     = 10    </P>
  <P> LY            = 6.5   </P>
  <P> HomModesZ     = 6     </P>
  <P> LZ            = 2.0   </P>
</PARAMETERS>
\end{lstlisting}

By default the opeartions associated with the harmonic expansions are performed
with the Matix-Vector-Multiplication (MVM) defined inside the code. The Fast
Fourier Transofrm (FFT) can be used to speed up the operations (if the FFTW
library has been compiled in ThirdParty, see the compilation instructions). To
use the FFT routines we need just to insert a flag in the solver information as
below:

\begin{lstlisting}[style=XMLStyle]
<SOLVERINFO>
  <I PROPERTY="EQTYPE" VALUE="UnsteadyAdvectionDiffusion" />
  <I PROPERTY="Projection" VALUE="Continuous"/>
  <I PROPERTY="HOMOGENEOUS" VALUE="2D"/>
  <I PROPERTY="DiffusionAdvancement" VALUE="Implicit"/>
  <I PROPERTY="AdvectionAdvancement" VALUE="Explicit"/>
  <I PROPERTY="TimeIntegrationMethod" VALUE="IMEXOrder2"/>
  <I PROPERTY="USEFFT" VALUE="FFTW"/>
</SOLVERINFO>
\end{lstlisting}

The number of homogenenous modes has to be even. The Quasi-3D apporach can be
created starting from a 2D mesh and adding one homogenous expansion or starting
form a 1D mesh and adding two homogeneous expansions. Not other options
available. In case of a 1D homogeneous extension, the homogeneous direction will
be the z-axis. In case of a 2D homogeneous extension, the homogeneous directions
will be the y-axis and the z-axis.

%%% Local Variables:
%%% mode: latex
%%% TeX-master: "../user-guide"
%%% End:
